\documentclass{bmstu}

\usepackage{biblatex}
\usepackage{array}
\usepackage{amsmath}

\addbibresource{inc/biblio/sources.bib}

\begin{document}

\makecourseworktitle
    {Информатика и системы управления}
    {Программное обеспечение ЭВМ и информационные технологии}
    {Разработка базы данных для хранения и обработки данных магазина одежды}
    {ИУ7-63Б}
    {И.~А.~Гринкевич}
    {А.~Л.~Исаев}
    {}
    {}
    
\setcounter{page}{3}

{\centering \chapter*{РЕФЕРАТ}}

Расчетно-пояснительная записка 36 с., 13 рис., 5 табл., 6 источн., 1 прил.

\noindent МАГАЗИН, ОДЕЖДА, ТОВАР, БАЗЫ ДАННЫХ, ВЕБ-ПРИЛОЖЕНИЕ, РЕЛЯЦИОННАЯ МОДЕЛЬ ДАННЫХ, SQL

Объектом разработки является база данных и приложение к ней.

Объектом исследования является время обработки запроса к веб-приложению.

Цель работы: разработка базы данных для хранения и обработки данных магазина одежды.

В результате выполнения работы была разработана база данных для хранения и обработки данных магазина одежды и веб-приложение, использующее эту базу данных.

Область применения результатов --- дальнейшее совершенствование приложения за счет системы рекомендаций товаров.

{\centering \maketableofcontents}

{\centering \chapter*{ПЕРЕЧЕНЬ СОКРАЩЕНИЙ И ОБОЗНАЧЕНИЙ}}

В настоящей расчетно-пояснительной записке к курсовой работе применяют следующие сокращения и обозначения:

\begin{table}[H]
\begin{tabular}{p{5cm}p{10.5cm}}
БД & База данных
\tabularnewline
ПО & Программное обеспечение
\tabularnewline
СУБД & Система управления базами данных
\tabularnewline
ER-диаграмма & Диаграмма <<сущность-связь>>
\tabularnewline
ID & Идентификатор
\tabularnewline
Use case диаграмма & Диаграмма вариантов использования
\tabularnewline
\end{tabular}
\end{table}

{\centering \chapter*{ВВЕДЕНИЕ}}
\addcontentsline{toc}{chapter}{ВВЕДЕНИЕ}

В современном мире информация является ключевым ресурсом для успешного ведения бизнеса~\cite{Business}. Эффективное управление данными становится все более важным для компаний, особенно в сфере розничной торговли. Магазины одежды не являются исключением, их успешная деятельность напрямую зависит от умения эффективно хранить, управлять разнообразными данными, связанными с продажами, ассортиментом, клиентами и многими другими аспектами бизнеса.

Целью данной курсовой работы является разработка базы данных для хранения и обработки данных магазина одежды.

Для достижения поставленной цели необходимо решить следующие задачи:
\begin{enumerate}
\item[1)] проанализировать предметную область;
\item[2)] сформулировать требования и ограничения к разрабатываемой базе данных;
\item[3)] формализовать информацию, хранимую в БД;
\item[4)] разработать структуру базы данных и определить ролевую модель в контексте БД;
\item[5)] избрать инструменты для разработки и реализовать спроектированную базу данных;
\item[6)] исследовать характеристики разработанного программного обеспечения.
\end{enumerate}

\chapter{Аналитическая часть}

\section{Анализ предметной области}

В настоящее время магазины одежды играют важную роль в розничной торговле России. Они предоставляют широкий ассортимент товаров и являются основными местами для покупки одежды и аксессуаров для населения. Среди наиболее популярных онлайн-магазинов одежды в России следует выделить такие платформы, как stockmann.ru и lamoda.ru. Они зарекомендовали себя как надежные и удобные ресурсы для поиска и приобретения модной одежды и обуви.

Однако, несмотря на популярность и широкий выбор товаров, часто пользователи сталкиваются с проблемами в использовании этих онлайн-платформ. Один из основных аспектов, на который обращают внимание, является перегруженный интерфейс и сложность использования. Слишком много информации, разнообразные дополнительные функции и не всегда интуитивно понятный дизайн могут создавать неудобства для покупателей. По этой причине одной из задач курсовой работы заключается в разработке базы данных, которая позволит магазину одежды предоставить простой и понятный функционал для своих клиентов.

Разработка простого и понятного функционала для магазина одежды позволит повысить удовлетворенность клиентов и укрепить позиции магазина на рынке.

\section{Требования к базе данных и приложению}

Приложение должно предоставлять пользователям возможность искать товары по следующим параметрам:
\begin{enumerate}
	\item[1)] пол человека;
	\item[2)] категория;
	\item[3)] бренд.
\end{enumerate}

Также должна быть возможность сортировки товаров по возрастанию или убыванию цены.

Пользователи должны иметь возможность зарегистрироваться, авторизоваться, просмотреть информацию о товарах и брендах, добавить и удалить товары из корзины, совершить заказ, просмотреть историю заказов.

Администратор должен иметь возможность просматривать информацию о заказах пользователей, изменять информацию о пользователях, заказах, товарах и брендах, а также загружать и удалять новые товары и бренды.

\section{Модели баз данных}

\subsection{Дореляционные базы данных}

Дореляционные базы данных - это тип баз данных, который не использует традиционные таблицы и связи, характерные для реляционных баз данных. Вместо этого они организуют данные в более гибких структурах, таких как документы, графы или временные ряды. Это позволяет более эффективно хранить и обрабатывать данные с учетом их специфики. Примерами дореляционных баз данных могут служить NoSQL базы данных, такие как MongoDB, CouchDB, Cassandra.

Преимущества дореляционных баз данных включают более гибкую структуру данных, что делает их подходящими для работы с неструктурированными или полуструктурированными данными. Они также могут быть более масштабируемыми в определенных сценариях, таких как обработка больших объемов данных.

Однако дореляционные базы данных могут потребовать более сложных запросов и могут не обеспечивать те же уровни нормализации данных, что и реляционные базы данных.

\subsection{Реляционные базы данных}

Реляционные базы данных - это тип баз данных, который использует табличную структуру для хранения данных и устанавливает связи между этими таблицами. Каждая таблица представляет собой набор записей с атрибутами, а связи между таблицами обеспечивают возможность объединения данных из разных таблиц при выполнении запросов.

Преимущества реляционных баз данных включают структурированный подход к хранению данных, что обеспечивает эффективность при работе с нормализованными данными~\cite{SQL}. Они также обладают мощным языком запросов SQL, который позволяет легко извлекать и модифицировать данные.

Однако реляционные базы данных могут столкнуться с проблемами при работе с неструктурированными данными или данными, требующими гибких схем.

\subsection{Постреляционные базы данных}

Постреляционные базы данных - это новый подход к хранению данных, который пытается объединить преимущества как дореляционных, так и реляционных баз данных. Данный подход предлагает гибкую схему данных, что делает БД такого типа способными к работе с разнообразными данными~\cite{PostSQL}. Постреляционные базы данных также предлагают улучшенные механизмы обработки больших объемов данных. Эти системы стремятся объединить гибкость дореляционных баз данных с мощью реляционных систем.

\section{Информация, подлежащая хранению в базе данных}

В разрабатываемой базе данных нужно будет хранить информацию о пяти сущностях: пользователь, бренд, товар, заказ и позиция заказа.
\begin{enumerate}
\item Пользователь. 
Информация о зарегистрированных пользователях: логин, пароль, имя, пол, роль.
\item Бренд. 
Информация о брендах товаров: название, год основания, идентификатор логотипа, название компании владельца.
\item Товар. 
Информация о товаре: категория, размер, цена, пол, идентификатор изображения, бренд, наличие.

\pagebreak

\item Заказ.
Информация о заказе: дата совершения, заказчик, цена, статус.
\item Позиция заказа.
Информация о позиции в заказе: заказ, товар, количество.
\end{enumerate}

Идентификаторы логотипа бренда и изображения товара представляют собой ID на сторонних сервисах для хранения изображений.

Так как база данных не имеет большого количества связей и хранимые в ней данные структурированы, для достижения поставленной цели была выбрана реляционная модель баз данных.

\section{ER-диаграмма сущностей базы данных}

На рисунке~\ref{img:er} показана диаграмма <<сущность-связь>> проектируемой базы данных в нотации Чена. 
Представлены пять сущностей и их свойства.

\includeimage
    {er}
    {f}
    {H}
    {1\textwidth}
    {ER-диаграмма}
    
\section{Пользователи приложения}

Пользователи могут иметь одну из трех ролей: посетитель, клиент и администратор.

Посетитель --- неавторизованный пользователь. 
Он может смотреть товары и бренды, авторизоваться и зарегистрироваться. На рисунке \ref{img:use-case-01} представлена use-case диаграмма для роли посетитель.

\includeimage
{use-case-01}
{f}
{H}
{.4\textwidth}
{Use-case диаграмма для роли посетитель}

Клиент --- зарегистрированный и авторизованный пользователь. 
Он имеет те же возможности что и посетитель (кроме регистрации), но в добавок он может добавить или удалить товары из корзины, посмотреть добавленные в корзину позиции, совершить заказ, просмотреть историю заказов. На рисунке \ref{img:use-case-02} представлена use-case диаграмма для роли клиент.

\includeimage
{use-case-02}
{f}
{H}
{.4\textwidth}
{Use-case диаграмма для роли клиент}

Администратор имеет те же возможности что и клиент, но вдобавок может просматривать информацию о заказах пользователей, изменять информацию о пользователях, заказах, товарах и брендах, а также загружать и удалять новые товары и бренды. На рисунке \ref{img:use-case-03} представлена use-case диаграмма для роли администратор.

\includeimage
{use-case-03}
{f}
{H}
{.4\textwidth}
{Use-case диаграмма для роли администратор}

\section*{Вывод из аналитической части}

По ходу выполнения аналитической части данной курсовой работы была проанализирована рассматриваемая предметная область, а так же сформулированы требования к разрабатываемой БД. Были рассмотрены модели баз данных и выбрана подходящая под поставленные задачи. Также была формализована информация, хранящиеся в БД и представлены диаграммы сущностей разрабатываемой базы данных и вариантов использования пользователями.

\chapter{Конструкторская часть}

\section{Описание сущностей базы данных}

На рисунке~\ref{img:db-diagram} представлена диаграмма базы данных.

\includeimage
    {db-diagram}
    {f}
    {H}
    {.9\textwidth}
    {Диаграмма базы данных}

В таблицах~\ref{tabular:users}--\ref{tabular:flights} описаны атрибуты таблиц базы данных.

\begin{table}[H]
\caption{Атрибуты таблицы <<Пользователи>>}
\label{tabular:users}
\begin{tabular}{|>{\raggedleft}p{4cm}|>{\raggedleft}p{3cm}|>{\raggedleft}p{8cm}|}
\hline
\textbf{Имя атрибута} & \textbf{Тип атрибута} & \textbf{Описание}
\tabularnewline
\hline
user\textunderscore id & целое & Идентификатор --- первичный ключ
\tabularnewline
\hline
user\textunderscore login & строка & Логин пользователя
\tabularnewline
\hline
user\textunderscore password & строка & Пароль пользователя
\tabularnewline
\hline
user\textunderscore name & строка & Имя пользователя
\tabularnewline
\hline
user\textunderscore sex & строка & Пол пользователя
\tabularnewline
\hline
user\textunderscore role & строка & Роль пользователя
\tabularnewline
\hline
\end{tabular}
\end{table}

\begin{table}[H]
\caption{Атрибуты таблицы <<Бренды>>}
\label{tabular:orders}
\begin{tabular}{|>{\raggedleft}p{4cm}|>{\raggedleft}p{3cm}|>{\raggedleft}p{8cm}|}
\hline
\textbf{Имя атрибута} & \textbf{Тип атрибута} & \textbf{Описание}
\tabularnewline
\hline
id & целое & Идентификатор --- первичный ключ
\tabularnewline
\hline
brand\textunderscore name & строка & Название бренда
\tabularnewline
\hline
founding\textunderscore year & целое & Год основания
\tabularnewline
\hline
logo\textunderscore id & целое & Идентификатор логотипа в стороннем сервисе
\tabularnewline
\hline
brand\textunderscore owner & строка & Название владельца бренда
\tabularnewline
\hline
\end{tabular}
\end{table}

\begin{table}[H]
\caption{Атрибуты таблицы-связки <<Товары>>}
\label{tabular:tickets_services}
\begin{tabular}{|>{\raggedleft}p{4cm}|>{\raggedleft}p{3cm}|>{\raggedleft}p{8cm}|}
\hline
\textbf{Имя атрибута} & \textbf{Тип атрибута} & \textbf{Описание}
\tabularnewline
\hline
id & целое & Идентификатор --- первичный ключ
\tabularnewline
\hline
category & строка & Категория товара
\tabularnewline
\hline
size & строка & Размер товара
\tabularnewline
\hline
price & целое & Цена
\tabularnewline
\hline
sex & строка & Пол товара
\tabularnewline
\hline
image\textunderscore id & целое & Идентификатор изображения товара в стороннем сервисе
\tabularnewline
\hline
brand\textunderscore id & целое & Идентификатор бренда --- внешний ключ
\tabularnewline
\hline
is\textunderscore available & логический тип & Наличие
\tabularnewline
\hline
\end{tabular}
\end{table}

\begin{table}[H]
\caption{Атрибуты таблицы <<Заказы>>}
\label{tabular:tickets}
\begin{tabular}{|>{\raggedleft}p{4cm}|>{\raggedleft}p{3cm}|>{\raggedleft}p{8cm}|}
\hline
\textbf{Имя атрибута} & \textbf{Тип атрибута} & \textbf{Описание}
\tabularnewline
\hline
id & целое & Идентификатор --- первичный ключ
\tabularnewline
\hline
commit\textunderscore date & дата & Дата совершения заказа
\tabularnewline
\hline
user\textunderscore id & целое & Идентификатор пользователя --- внешний ключ
\tabularnewline
\hline
price & целое & Суммарная цена товаров в заказе
\tabularnewline
\hline
current\textunderscore status & строка & Статус заказа
\tabularnewline
\hline
\end{tabular}
\end{table}

\begin{table}[H]
\caption{Атрибуты таблицы <<Позиции заказов>>}
\label{tabular:flights}
\begin{tabular}{|>{\raggedleft}p{4cm}|>{\raggedleft}p{3cm}|>{\raggedleft}p{8cm}|}
\hline
\textbf{Имя атрибута} & \textbf{Тип атрибута} & \textbf{Описание}
\tabularnewline
\hline
id & целое & Идентификатор --- первичный ключ
\tabularnewline
\hline
order\textunderscore id & целое & Идентификатор заказа --- внешний ключ
\tabularnewline
\hline
item\textunderscore id & целое & Идентификатор товара --- внешний ключ
\tabularnewline
\hline
amount & целое & Количество товаров
\tabularnewline
\hline
\end{tabular}
\end{table}

\section{Описание ограничений целостности базы данных}

Для обеспечения целостности базы данных, на хранящуюся в ней информацию должны быть наложены ограничения.

Ограничения для таблицы <<Пользователи>>. 
Идентификатор --- уникальное положительное число, первичный ключ. 
Логин, пароль, имя, пол и роль пользователя не должны отсутствовать.

Ограничения для таблицы <<Бренды>>. 
Идентификатор --- уникальное положительное число, первичный ключ. 
Имя бренда, год основания, идентификатор логотипа и название владельца не должны отсутствовать, также год основания должен находиться в интервале от 1500 до 2024.

Ограничения для таблицы <<Товары>>. 
Идентификатор --- уникальное положительное число, первичный ключ. 
Идентификатор бренда --- положительное число, внешний ключ, не должен отсутствовать. 
Категория, размер, цена, пол, идентификатор изображения, наличие не должны отсутствовать, также цена должна быть положительной.

Ограничения для таблицы <<Заказы>>. 
Идентификатор --- уникальное положительное число, первичный ключ. 
Идентификатор пользователя --- положительное число, внешний ключ, не должен отсутствовать. 
Дата совершения, цена и статус не должны отсутствовать. Цена должна быть положительной.

Ограничения для таблицы <<Позиции заказов>>. 
Идентификатор --- уникальное положительное число, первичный ключ. 
Идентификаторы заказа и товара --- положительные числа, внешние ключи, не должны отсутствовать. 
Количество не должно отсутствовать и быть положительным.

\section{Описание проектируемой функции на уровне базы данных}

Пользователь может смотреть позиции, добавленные в корзину, для этой цели удобно реализовать функцию, которая по идентификатору пользователя будет возвращать позиции в корзине.

На рисунке~\ref{img:function} представлена схема проектируемой функции.

\includeimage
    {function}
    {f}
    {H}
    {0.7\textwidth}
    {Функция, возвращающая позиции в корзине}

В функции используется переменная basket\textunderscore id --- идентификатор заказа пользователя со статусом <<корзина>>.

\section{Описание ролевой модели на уровне базы данных}

Для работы с базой данных было предусмотрено создание трех ролей: посетитель, клиент и администратор.

Посетитель имеет доступ к просмотру информации о таких объектах базы данных, как товар и бренд.

Клиент имеет доступ к просмотру информации о товарах, брендах, заказах и позициях в заказах.

Администратор имеет возможность просматривать, добавлять, изменять и удалять информацию о любом объекте БД.

\section*{Вывод из конструкторской части}

В ходе выполнения конструкторской части данной курсовой работы было представлено описание сущностей разрабатываемой базы данных и представлена соответствующая диаграмма; описаны атрибуты таблиц рассматриваемой БД; описаны ограничения накладываемые на информацию хранимую базе данных для обеспечения ее целостности; описана проектируемая функция и ролевая модель на уровне базы данных.

\chapter{Технологическая часть}

\section{Средства реализации}

Для разработки веб-сервера в рамках данного проекта был выбран язык программирования Golang~\cite{Go}, и данный выбор был обоснован по следующим причинам:

\begin{enumerate}
	\item Простота и легкость в освоении: Go обладает синтаксисом, который легко понимать и осваивать, что способствует быстрой адаптации разработчиков к этому языку программирования;
	\item Высокая производительность: Go предоставляет возможность создания высокопроизводительных приложений за счет низкого уровня абстракции;
	\item Разнообразная экосистема: Go обладает обширной библиотекой стандартных пакетов и доступом к множеству сторонних библиотек и фреймворков;
	\item Надежность: Go был создан Google для разработки надежных и стабильных приложений, что обеспечивает его высокую надежность.
\end{enumerate}

В качестве системы управления базами данных была выбрана PostgreSQL по причине того, что данная СУБД часто используется в проектах на Golang, в следствие чего существует большое разнообразие библиотек и документации для работы с PostgreSQL~\cite{Postgres} на Go.

\pagebreak

\section{Реализация сущностей базы данных}

В листингах~\ref{lst:user.go}--\ref{lst:orderItem.go} показана реализация сущностей базы данных.

\includelisting
{user.go}{Cущность <<Пользователь>>}

\includelisting
{brand.go}{Cущность <<Бренд>>}

\includelisting
{item.go}{Cущность <<Товар>>}

\pagebreak

\includelisting
{order.go}{Cущность <<Заказ>>}

\includelisting
{orderItem.go}{Cущность <<Позиция заказа>>}



\section{Реализация ограничений целостности базы данных}

В листингах~\ref{lst:constraints_user.sql}--\ref{lst:constraints_orderItems.sql} показана реализация ограничений целостности базы данных (ограничения задаются при создании таблиц).

\includelisting
{constraints_user.sql}{Ограничения для таблицы <<Пользователи>>}

\pagebreak

\includelisting
{constraints_brand.sql}{Ограничения для таблицы <<Бренды>>}

\includelisting
{constraints_item.sql}{Ограничения для таблицы <<Товары>>}

\includelisting
{constraints_ordering.sql}{Ограничения для таблицы <<Заказы>>}

\pagebreak

\includelisting
{constraints_orderItems.sql}{Ограничения для таблицы <<Позиции заказов>>}

\section{Реализация функции на уровне базы данных}

В листингах~\ref{lst:db_function.sql}--\ref{lst:db_function_continue.sql} показана реализация функции на уровне базы данных, которая по идентификатору пользователя возвращает позиции в его корзине. 

\includelisting
{db_function.sql}{Функция, возвращающая позиции в корзине пользователя по его идентификатору}

\includelisting
{db_function_continue.sql}{Функция, возвращающая позиции в корзине пользователя по его идентификатору (продолжение)}

\section{Реализация ролевой модели на уровне базы данных}

В листингах~\ref{lst:db_roles.sql}--\ref{lst:db_roles_continue.sql} представлена реализация ролевой модели на уровне базы данных.

\includelisting
{db_roles.sql}{Ролевая модель на уровне базы данных}

\includelisting
{db_roles_continue.sql}{Ролевая модель на уровне базы данных (продолжение)}


\section{Примеры работы}

Для демонстрации работы разработанного ПО были выполнены следующие запросы:

\begin{enumerate}
	\item регистрация пользователя;
	\item авторизация пользователя;
	\item получение списка товаров;
	\item добавление товара в корзину;
	\item попытка создать новый товар клиентом.
\end{enumerate}

На рисунках~\ref{img:testRegister}--\ref{img:testItem} представлены примеры запросов к серверу и результаты их обработки.

\includeimage
{testRegister}
{f}
{H}
{0.8\textwidth}
{Пример регистрации пользователя}

\includeimage
{testLogin}
{f}
{H}
{0.8\textwidth}
{Пример авторизации пользователя}

\includeimage
{testItems}
{f}
{H}
{0.8\textwidth}
{Пример запроса на получение списка товаров}

\includeimage
{testBasket}
{f}
{H}
{0.8\textwidth}
{Пример запроса на добавление товара в корзину}

\includeimage
{testItem}
{f}
{H}
{0.8\textwidth}
{Пример запроса на создание нового товара клиентом}


\section*{Вывод из технологической части}

В ходе выполнения технологической части данной курсовой работы были выбраны средства реализации разрабатываемого ПО; представлены листинги реализаций сущностей и ограничений целостности БД; представлены листинги реализации функции и ролевой модели на уровне базы данных, а также продемонстрированы примеры работы программы на различных запросах.

\chapter{Исследовательская часть}

\section{Технические характеристики устройства}

Технические характеристики устройства, на котором были проведены исследования:

\begin{enumerate}
\item[1)]
операционная система Windows 11 Home x64;
\item[2)]
оперативная память 8 ГБ;
\item[3)]
процессор AMD Ryzen 5 4500U with Radeon Graphics @ 2.38 ГГц.
\end{enumerate}

Измерение времени ответов сервера на запросы проводилось с помощью инструмента для нагрузочного тестирования Locust~\cite{Locust}.

\section{Время выполнения запроса}

Для исследования зависимости времени выполнения запросов от количества пользователей были проведены 2 теста.

Пользовательский сценарий первого теста:

\begin{enumerate}
	\item[1)]
	POST(<</login>>)
	\item[2)]
	GET(<</item>>)
	\item[3)]
	GET(<</brand>>)
\end{enumerate}

В первом тесте количество пользователей росло от 1 до 1000 с шагом 15. Таблицы <<Товары>> и <<Бренды>> содержат 5000 и 1000 строк соответственно. На рисунке ~\ref{img:hard1} представлены результаты теста.

\includeimage
{hard1}
{f}
{H}
{1\textwidth}
{Графики зависимости количества запросов в секунду и времени ответа от количества пользователей для первого теста}

Пользовательский сценарий второго теста:

\begin{enumerate}
	\item[1)]
	POST(<</login>>)
	\item[2)]
	GET(<</basket>>)
	\item[3)]
	GET(<</basket>>)
\end{enumerate}

В данном тесте количество пользователей росло от 1 до 1000 с шагом 15. Таблица <<Заказы>> содержит 10000 строк. На рисунке ~\ref{img:hard2} представлены результаты теста.

\includeimage
{hard2}
{f}
{H}
{1\textwidth}
{Графики зависимости количества запросов в секунду и времени ответа от количества пользователей для второго теста}

\section*{Вывод из исследовательской части}

В ходе выполнения исследовательской части данной курсовой работы было выявлено, что при запросе к таблице, содержащей 1000-5000 строк, максимальное количество запросов, которое может обработать сервер без задержки равно примерно 550, после превышения данного значения время ответа сервера начинает заметно расти. При запросе к таблице, содержащей 10000 строк, максимальное количество запросов, которое может обработать сервер без задержки равно примерно 180, после превышения данного значения время ответа сервера начинает заметно расти, а после превышения значения 190 начинает расти количество ошибок сервера в ответ на запрос в связи с перегруженностью.

{\centering \chapter*{ЗАКЛЮЧЕНИЕ}}
\addcontentsline{toc}{chapter}{ЗАКЛЮЧЕНИЕ}

В результате выполнения данной курсовой работы была разработана база данных для хранения и обработки данных магазина одежды.

В ходе выполнения курсовой работы были достигнуты следующие задачи:
\begin{enumerate}
	\item[1)] проанализировать предметную область;
	\item[2)] сформулировать требования и ограничения к разрабатываемой базе данных;
	\item[3)] формализовать информацию, хранимую в БД;
	\item[4)] разработать структуру базы данных и определить ролевую модель в контексте БД;
	\item[5)] избрать инструменты для разработки и реализовать спроектированную базу данных;
	\item[6)] исследовать характеристики разработанного программного обеспечения.
\end{enumerate}

{\centering \printbibliography[title=СПИСОК ИСПОЛЬЗОВАННЫХ ИСТОЧНИКОВ,heading=bibintoc]}

{\centering \chapter*{ПРИЛОЖЕНИЕ А}}
\addcontentsline{toc}{chapter}{ПРИЛОЖЕНИЕ А Презентация}
\center{\textbf{Презентация}}

\end{document}
